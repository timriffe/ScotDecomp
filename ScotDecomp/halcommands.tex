
% new commands defined

%bold faced letters
\newcommand{\bo}[1]{{\bf #1}}

\newcommand{\bm}[1]{\mbox{\boldmath $#1$}}
\newcommand{\kron}{\otimes}
\renewcommand{\vec}{\mbox{vec} \,}

%begin and end equations
\newcommand{\be}{\begin{equation}}
\newcommand{\ee}{\end{equation}}

%begin and end equations without numbers
\newcommand{\bd}{\begin{displaymath}}
\newcommand{\ed}{\end{displaymath}}

%begin and end equation arrays
\newcommand{\bea}{\begin{eqnarray}}
\newcommand{\eea}{\end{eqnarray}}

%begin and end equation arrays without numbers
\newcommand{\beastar}{\begin{eqnarray*}}
\newcommand{\eeastar}{\end{eqnarray*}}

%begin and end matrices
\newcommand{\bmat}[1]{\left(\begin{array}{#1}}
\newcommand{\emat}{\end{array}\right)}

%equation numbers
\newcommand{\enum}[1]{\label{eq:#1}}

%derivatives and elasticities
\newcommand{\der}[2]{{d #1 \over d #2}}
\newcommand{\elas}[2]{{\varepsilon #1 \over \varepsilon #2}}


%partial derivatives and second partial derivatives
\newcommand{\pder}[2]{{\partial #1 \over \partial #2}}
\newcommand{\secpderij}[3]{{\partial^2 #1 \over \partial #2 
            \partial #3}}
\newcommand{\tsecpderij}[3]{\partial^2 #1 / \partial #2 
            \partial #3}  %text second partial

\newcommand{\secpderii}[2]{{\partial^2 #1 \over \partial #2^2}}
\newcommand{\tsecpderii}[2]{\partial^2 #1 / \partial #2^2}
%   text second partial

%matrix transpose symbol
\newcommand{\tr}{{\mbox{\tiny \sf T}}}

%matrix diagonal symbol
%\newcommand{\diag}{\mbox{diag} \,}
\newcommand{\diag}{\mathcal{D}}

\newcommand{\trace}{\mbox{trace}}

\newcommand{\dg}{{\rm{dg}}}

% \mc{number of columns}{position}{item}
\newcommand{\mc}[3]{\multicolumn{#1}{#2}{#3}}
% center a single item in a table
\newcommand{\cent}[1]{\mc{1}{c}{#1}}

\newcommand{\mcal}[1]{\mathcal{#1}}


\newcommand{\str}[1]{\rule{0in}{#1}}
\newcommand{\tbar}[1]{\left. \rule{0in}{#1} \right|}





\newcommand{\rind}{\noindent\hangindent=20pt\hangafter=1}

%%%%%%%%%
% margin adjustments
% these are for US letter size paper; should be adjusted for A4
%%%%%%%%

\textwidth=6.5in
 \oddsidemargin=0in
 %\evensidemargin=0.125in
 %\evensidemargin=0.5in
 \textheight=9.0in
 \topmargin=0.0in
 \headheight=0.0in
 \headsep=0.0in

%%%%%%%%%%%%
% end of margin adjustments
%%%%%%%%%%%% 

%%%%%% stuff from here on not used much %%%%%

% \newtheorem{theorem}{Theorem}[section]
%\newtheorem{theorem}{Theorem}

%\newtheorem{example}{Example}[section]
\newtheorem{example}{Example}

    \newcommand{\exbegin}[1]{\begin{itemize} \item[~]

    \begin{example}{\bf #1} \begin{rm} }
    \newcommand{\exend}{\end{rm}\end{example}
    \end{itemize}}
\newcommand{\scp}[1]{\langle #1 \rangle}
\setcounter{secnumdepth}{3}
\setcounter{tocdepth}{4}

\newtheorem{newfig}{Figure}
\newcommand{\figbegin}{\begin{newfig} \begin{rm} }
\newcommand{\figend}{\end{rm} \end{newfig}}
% commands to change float restrictions
\renewcommand{\topfraction}{0.99}
\renewcommand{\bottomfraction}{0.99}
\renewcommand{\textfraction}{0}
\renewcommand{\floatpagefraction}{0}

\newcommand{\mathbold}[1]{\mbox{\boldmath $\bf#1$}}
% Shultis, Latex Notes, p. 54


%Commands for biblist
\newenvironment{biblist}{\begin{list}{}{\setlength{\leftmargin}{0.15in}% 
\setlength{\itemindent}{-0.15in} \setlength{\parsep}{0.in}%
\setlength{\itemsep}{0.1in}}}{\end{list}} 