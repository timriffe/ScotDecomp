\documentclass[12pt,oneside,a4paper]{article} % for sharing
\usepackage{apacite}
\usepackage{appendix}
\usepackage{amsmath}
\usepackage{amsthm}
\usepackage{multirow}
\usepackage{amssymb} % for approx greater than
\usepackage{caption}
\usepackage{placeins} % for \FloatBarrier
\usepackage{graphicx}
\usepackage{subcaption}
\usepackage{longtable}
\usepackage{setspace}
\usepackage{booktabs}
\usepackage{tabularx}
\usepackage{xcolor,colortbl}
\usepackage{chngpage}
\usepackage{natbib}
\bibpunct{(}{)}{,}{a}{}{;} 
\usepackage{url}
\usepackage{nth}
\usepackage{authblk}
\usepackage[most]{tcolorbox}
\usepackage[normalem]{ulem}
\usepackage{amsfonts}

% scraped out necessary stuff from hal's loghead file
\input{halcommands.tex}

% columns for longtable
\newcolumntype{C}[1]{>{\centering\let\newline\\\arraybackslash\hspace{0pt}}m{#1}}
\newcolumntype{L}[1]{>{\raggedright\let\newline\\\arraybackslash\hspace{0pt}}m{#1}}
\usepackage{arydshln} % Dashed lines in matrices


\usepackage[margin=1in]{geometry}
%\doublespacing % for review

% line numbers to make review easier
%\usepackage{lineno}
%\linenumbers

%\usepackage{soul}% for \st{}

%%%%%%%%%%%%%%%%%%%%%%%%%%%%%%%%%%%%%%%%%%%%%%%%%%%%%%%%%%%%%%%%%%%%%%%%%%%%%%
% for section 4 math environments
\theoremstyle{definition}
\newtheorem{definition}{Definition}[section]
\newtheorem{theorem}{Theorem}[section]
\newtheorem{proposition}{Proposition}[section]
\newtheorem{corollary}{Corollary}[proposition]
\newtheorem{remark}{Remark}[section]

%%%%%%%%%%%%%%%%%%%%%%%%%%%%%%%%%%%%%%%%%%%%%%%%%%%%%%%%%%%%%%%%%%%%%%%%%%%%%%

\newcommand\ackn[1]{%
  \begingroup
  \renewcommand\thefootnote{}\footnote{#1}%
  \addtocounter{footnote}{-1}%
  \endgroup
}

% Affiliations in small font size
\renewcommand\Affilfont{\small}

\defcitealias{HMD}{HMD 2016}

% junk for longtable caption
\AtBeginEnvironment{longtable}{\linespread{1}\selectfont}
\setlength{\LTcapwidth}{\linewidth}

% sort van Raalte properly
% #1: sorting key, #2: prefix for citation, #3: prefix for bibliography
\DeclareRobustCommand{\VAN}[3]{#2} % set up for citation

%%%%%%%%%%%%%%%%%%%%%%%%%%%%%%%
\begin{document}


\title{The changing contribution of socioeconomic deprivation to variance in age at death}
%\author{author(s) redacted}
\author[1]{Rosie Seaman\thanks{seaman@demogr.mpg.de}}
\author[1]{Tim Riffe}
\author[2]{Hal Caswell}

\affil[1]{Max Planck Institute for Demographic Research, Rostock, Germany}
\affil[2]{University of Amsterdam, Netherlands}

\maketitle

\begin{abstract}
Mortality inequalities demonstrate a double burden: the most deprived socioeconomic
groups experience the lowest average age of death and the highest variation in
age at death. Two processes generate variation in age at death: individual
stochasticity (within-group variance) and heterogeneity (between-group
inequality). Limited research has evaluated how these two components have
changed over time. We address this research gap by using population and
mortality data for the entire population of Scotland stratified by a validated
measure of area-level deprivation that covers the time period 1981-2011. The
most deprived areas have experienced stagnating or slight increasing variance in
age at death and the least deprived areas have experienced decreasing variance.
This is consistent with the existing literature demonstrating that there is not
simply a social gradient for variation in age at death but that socioeconomic groups have experienced diverging trends. While the contributions from within-group inequality did not change over the study period, the contributions from between-group inequality increased, indicating relatively greater importance of area level mortality differences for total variation in age at death.
\end{abstract}

\section{Background}
The association between socioeconomic inequality and mortality is traditionally
evidenced by life expectancy comparisons. The most deprived populations
experience the lowest average age of death, and the least deprived populations
experience the highest. Studies have further demonstrated that the most deprived
populations also demonstrate the highest level of variation in age at death when
measuring socioeconomic inequality by income, education, or occupation
\citep{Broennum-Hansen2017,Raalte2011,Sasson2016,Raalte2014}. The patterning of these two dimensions of mortality represents a double burden of inequality and routine monitoring of both is important for evaluating the extent to which improving average mortality and reducing inequalities are being achieved simultaneously.

ADD PARAGRAPH
Why do we think life disparity is a burden. 1) life planability, the ability to
anticipate and adapt to one's future, whether consious or not constrains and
conditions important life transitions and investments in health and education in
the present. Make sense of that. 2) because it's mostly due to excess premature
mortality. All premature mortality is bad, but within that we can still say that
some is excessive and some is to be expected, namely. 3) governments would
prefer to have lower variance-- see Alyson's argument. 

How variance can be misinterpreted: there is early and late mortality, both of
which add to variance, and of which we consider early to be bad and late to be
good. It is perfectly possible for select subgroups to experience increases in
variance due to a long right tail in the deaths distribution, but such cases are
not. Respond to people who think this might be age discrinating in some
thoughtful engaging way.
------------------

A number of highly correlated indices measure disparity in age at
death \citep{Raalte2013}, where low disparity indicates less uncertainty in age
at death for individuals. For societies, relatively low and decreasing disparity
means that the risk of premature death is being reduced for the population which, in turn, may reflect a more efficient social security system with higher welfare redistribution 
\citep{Raalte2012,Bambra2011,Popham2013}.\footnote{Disparity comes from both
premature deaths and deaths at very old ages, and theoretically the latter could
drive increasing variance, but this is not the case for Scotland or any of its
deprivation quantiles.} We use lifetable variance to quantify disparity for
two reasons: 1) There is a well-known analytic method to decompose variance
into within and between group components. 2) We can transform variance into
standard deviations to report results in year units.
 
Two processes underly the total variance in age at death: individual
stochasticity (within-group variance) and heterogeneity (between-group
inequality). Within-group variance tends to arise from differences due to random
demographic processes. The lifetable assumes that every individual, at the same age, is subject
to the same schedule of mortality rates, such that any
variance in age of death can be interpreted as individual stochasticity.
Within-group variance can also be due to unaccounted for subgroup heterogeneity.
For example males and females have different mortality schedules. Aggregating
a lifetable over both sexes increases within-group variance due to induced
between group heterogeneity, even if males and females have identical
within-group variance. Within-group variance is theoretically always inflated
due to heterogeneity on unmeasured characteristics of the population. For
example, we cannot stratify our lifetables on other characteristics such as
marital, employment, or diabetes status. These and similar individual
characteristics that one could hypothetically group on will likely account for a
non-trivial fraction of within group variance.
%It is also
%possible that a lifetable could be produced for populations that are
% heterogeneous but without knowing the reasons why the mortality of the populations are disparate and being unable to produce stratified lifetables.

Between-group lifespan inequality arises when individuals at the same age are
subject to different mortality rates, which may be due to exposures to
different social, economic, or environmental contexts \citep{Hartemink2017}.
\citet{Raalte2012} estimates the contribution of educational inequalities (the
between-group component) to the total variance in age at death for 11 individual European countries. For males in Sweden the between-group component accounted for 1.7\% of the total variance in age at death but for males in the Czech Republic it accounted for 10.9\% of total variance in age at death. The between-group component was higher in the Czech Republic because the age distributions of death, stratified by education, are more disparate than in Sweden. 
%Noted was the distinct age distribution for the least educated males in the
% Czech Republic. 
\citet{Raalte2012} used data aggregated over 1990 to 2000, such
that time trends for the between-group and within-group components could not be assessed. 
% check- is that in said paper, ir did it come from Hendi?
In addition, it recognised that education may be a problematic socioeconomic
measure for studying trends in between-group components due to
changes in educational composition and the meanining of education attainment
\citep{Hendi2015}.
A further limitation when stratifying data by education, or other socioeconomic
states that are theoretically acquired over the life course, is that researchers
may need to left-truncate data at some age because education is aqcuired
over the life course. 
%(although
%\citet{Broennum-Hansen2017} used an indicator of disposable household income
%that does not require age truncation, results were still reported as age
% conditional).
Area level measures of relative deprivation, although not interchangeable with individual level socioeconomic measures, can help to over come these practical limitations and also have a number of theoretical advantages.
 
Relative deprivation is the idea that not having enough material cultural
or social resources to participate in the accepted way of life in one's societal
context. This concept of deprivation has been found to be as important for
health as the absolute minimum requirements for survival (absolute poverty)
\citep{Townsend1987,Carstairs1989,Kearns2000,Kawachi2002}. When stratifying
groups on quantiles, relative measures of deprivation likely have a
more consistent interpretation over time than do absolute measures; although
absolute levels of poverty in Scotland have improved there is always a notional
most deprived fraction of the population. 

ROSIE: think on this. Is it pragmatic to assign postal code area deprivation to
individuals because it is derived from those same individuals? And because
these areas are small and presumably more homogenously composed than are
alternative large areas. We know there is segrations (Sabater.. bla bla)

 Second, assigning
individuals to an area level measure of deprivation based on their post-code is pragmatic: Home
address is routinely collected across all stages of the life course and
passively by a range of health services, while measures of income, occupation,
or education are not. 

Third, area of residence is less age dependent than educational attainment or
other similar SES categories.
Therefore, age truncation is not necessary and few members of the population
are excluded.\footnote{Students, the unemployed, and retirees for example are
included. Military bases and offshore drilling stations are not assigned
Carstairs scores, and deaths ocurring in these places are also excluded.}

AH HA- confusion is due to area measures derived from people in area vs
contextual attributes of the area such as number of hospitals, distance to bla
bla. Clarify following sentence wrt first sentence in paragraph.

 Fourth, area
level indicators are also derived from routinely collected data which allows for
regular updates and makes them a powerful tool for governments assessing how best to distribute resources. Finally, area level deprivation may have an influence on risk of death independent of individual level socioeconomic circumstance \citep{Carstairs1989,Macintyre2002,Tunstall2011}. Although empirical results are mixed it remains valuable to understand that mortality outcomes are not only driven by characteristics of individuals but also by the collective and contextual characteristics of areas \citep{Macintyre2002}.

We contribute to the mortality inequalities literature by measuring the changing contributions from within and
between-group components to variance in age at death. Data are centered on
Census years, ensuring the most robust population estimates available.
Populations in each postal code are aggregated base on population-weighted
quintiles of the Carstairs score distribution. Death counts are
matched to postal codes based on place of usual residence and then aggregated
on Carstairs quintiles. The data include the whole population of Scotland (ca 5
million persons) and cover the time period between 1981 and 2011.

\section{Data and methods}

\subsection{Area level deprivation}
Census population estimates and mortality data\footnote{Mortality data used
came from 1980-1982, 1990-1991, 2000-2002 and 2010-2012 to increase the
number of events centered around each census. 1990-1991 is just a two-year
numerator sample due to geographical boundary changes occurring in 1990.
Corresponding Census population estimates are adjusted accordingly.} by single
year of age and sex for each part-postcode (zip-code) sector in Scotland were
obtained via a commissioned request to National Records of Scotland. There are
around 1,012 part-postcode sectors in Scotland at each Census year each with an
average population size of 5,000 individuals\footnote{MAKE TABLE: 1981 total
number of postcode sectors=1010, Mean population (SD) =4982.47 (1178.53) 1991 total number of postcode sectors=1001 mean population (SD) 4993.02 (1653.67). 2001 total number of postcode sectors=1010 mean population (SD) =5011.89 (1542.42). 2011 total number of postcode sectors=1012 mean population (SD) =5232.61 (1568.05)}.
 
Population-weighted quintiles (each 20\% of the population) were created by
aggregating the 1,012 part-postcode sectors ordered on Carstairs score of
deprivation. (INTEND TO ADD IN 4 MAPS OF SCOTLAND HERE SHOWING THE GEOGRAPHICAL OUTLINES OF EACH POSTCODE SECTOR AND THE QUINTILE OF DEPRIVATION IT WAS ASSIGNED TO AT EACH CENSUS YEAR). The Carstairs score is a z-score for each part-postcode sector
that is derived from four individual-level census variables: overcrowding, male
unemployment, low social class, and car ownership. The Carstairs Score
(z-score) reflects the material resources that provide the means to access the
goods, services, amenities, and physical environment seen as expected
in society \citep{Carstairs1989}. This means the Carstairs score is a method of
capturing relative deprivation at the contextual level (cite). Scores range from
XXX to XXX and are centered on zero, with higher scores indicating relatively
higher deprivation than the national level.

Deaths and Census population denominators were used to construct complete
lifetables for each deprivation quintile, centered on each Census year, for
males and females seperately. The Human Mortality Database
Methods Protocol was used to extrapolate age specific mortality rates from
ages 85 to 110 \citep{Wilmoth2017}.\footnote{Specifically, we apply equations (53) and
(54) from the HMD protocol v6, modified to use information from ages 75+
rather than 80+.} From the complete lifetables we compute remaining
life expectancy and the conditional variance and standard deviation of
the remaining lifespan distribution. Lifetable standard deviations are a common
measure of the variability applied to the distribution of age at death \citep{Raalte2013}. Lifetable variance is calculated and decomposed into within- and between-group components using Markov chain methods \citep{Caswell2001}\citep{Caswell2009}\citep{Caswell2014}. Details of the within-group and between-group component calculations are given in the supplementary files.


\subsection{Variance decomposition}
Lifetable variance is calculated and decomposed into within- and between-group
components using Markov chain methods \citep{Caswell2001}\citep{Caswell2009}\citep{Caswell2014}. From the lifetable, we extract
conditional single-age death probabilities, $q_x$, and take its complement,
$p_x$. We then calculate the survival matrix for the $i^{th}$
quintile, $\bo U_i$ as:
\begin{equation}
\bo U_i = 
\begin{bmatrix}
    0     & \hdots  & \hdots &  \hdots  & 0 \\
    p_{1} &   &    &    &  \vdots \\
    0 & \ddots &   &   & \vdots \\
    \vdots & & \ddots & & 0\\
   0 &  \hdots & 0 & p_{\omega-1}  & p_{\omega}
\end{bmatrix}
\end{equation}
Conditional remaining survivorship is calculated as:
\begin{equation}
\mathbf{N}_i = (\mathbf{I} - \mathbf{U}_i )^{-1} \quad .
\end{equation}
$\mathbf{N}_i$ ends up being 0s in the upper triangle, and conditional remaining
survivorship in columns descending from the subdiagonal. The moments of longevity for individuals in group $i$ are $\bm \eta_1^{(i)}$ and $\bm \eta_2^{(i)}$. 
\begin{equation}
\bm \eta_1^{(i)} = (1^\tr \bm N_i)^\tr
\end{equation}
The second moment is defined as:
\begin{equation}
\bm \eta_2^{(i)} = \left[ 1^\tr \bm N_i (2 \bm N_i - \bm I)\right]^\tr
\end{equation}
The vectors with the means and variances, for group $i$, are
\bea
E(\bm \eta^{(i)}) &=& \bm \eta_1^{(i)} \\
V(\bm \eta^{(i)}) &=& \bm \eta_2^{(i)} - \left[ \bm \eta_1^{(i)} \ \circ \bm \eta_1^{(i)} \right]
\eea

To carry out calculations we procede by creating vectors that contain the
combined age and stage specific values
\bea
E(\tilde{\bm \eta}) = \bmat{c}
E(\bm \eta^{(1)}) \\
\vdots\\
E(\bm \eta^{(g)})
\emat
\eea
and a similar vector for variances $V(\tilde{\bm \eta})$. The tilde indicates
that these combine both age and quintile values, with length = $g \omega$.

The next step is to calculate the means and variances of remaining longevity, at
each age $x$, within each group, as follows.
\bea
E(\bm \eta(x)) &=& \left( \bo I_g \kron \bo e_x^\tr \right) E(\tilde{\bm \eta}) \qquad x=1,\ldots,\omega  \\[1ex]
V(\bm \eta(x)) &=& \left( \bo I_g \kron \bo e_x^\tr \right) V(\tilde{\bm \eta}) \qquad x=1,\ldots,\omega
\eea
where $\bo e_x$ is a vector of length $\omega$ with a 1 in the $x$th position and zeros elsewhere. The resulting vectors here are of dimension $g \times 1$.

At age $x$ the cohort consists of a mixture of the $g$ different groups ($g=5$
for quintiles, 10 for deciles) with mixing distribution
$\bm \pi(x)$ generated by the differential survival of groups within the cohort.


The mixing distribution \bm \pi(x) at age $x$ is a vector of dimension $g\times 1$ , which sums to 1. It is obtained from the distribution of groups in an initial cohort. Since quintiles are by definition equally distributed, it would seem that the initial cohort should be evenly distributed. Some other distribution could be used if desired. 

Let \bm \pi(0) be the initial mixing distribution, and let $\bm \eta^{(i)}(0)$ be the initial cohort age distribution in group $i$. Then 
\begin{equation}
\bo n^{(i)} (0) = \bo{e}_i \pi_i(0)
\end{equation}
(i.e., a vector with $\pi_i (0)$ in the first entry and zeros elsewhere. For an evenly distributed cohort, $\pi_i (0)=1/g$.) 

Project each group with its appropriate survival matrix 
\begin{equation}
\bo n^{(i)} (x) = \bo U_i^{x}\bo n^{(i)}(0)  \qquad i=1,\ldots,g\
\end{equation}

add up the entries
\begin{equation}
N^{(i)}(x)=\bo 1_{\omega}^{\tr}\bo n^{(i)}(x) \qquad i=1,\ldots,g\
\end{equation}

and create $\bm \pi(x)$ by putting these into a vector and normalizing it to sum to 1
\bea
\bm \pi(x) = \bmat{c}
N^{(1)}(x) \\
\vdots\\
N^{(g)}(x)
\emat
\frac{1}{\sum_{i}N(i)(x)}
\eea


At age $x$ remaining life expectancy for the total population is
\bea
E (\eta_x) &=& E_{\bm \pi(x)} \left[ E(\bm \eta(x)) \right] \\[1ex]
&=& \bm \pi(x)^\tr E(\bm \eta(x)) \\[1ex]
&=& \left( \bm \pi(x)^\tr \kron \bo e_x \right) E(\tilde{\bm \eta}) \qquad x=1,\ldots,\omega
\eea
Notice that $\eta_x$ is a scalar. The remaining life expectancy at age $x$ is a
simple average weighted by the mixing distribution.

The variance in $\eta_x$ is
\be
V(\eta_x) = V_{\rm within} + V_{\rm between}
\ee
with
\bea 
V_{\rm within} &=& E_{\bm \pi(x)} \left[ \str{2.5ex}V \left( \bm \eta(x) \right) \right]  \\[1ex]
&=&
\bm \pi(x)^\tr V(\bm \eta(x) )\\[1ex]
&=& \left( \bm \pi(x)^\tr \kron \bo e_x^\tr \right) V(\tilde{\bm \eta}(x)) 
\eea
and
\bea
V_{\rm between} &=& V_{\bm \pi(x)} \left[ \str{2.5ex} E \left( \bm \eta(x) \right) \right] \\
&=&
\bm \pi(x)^\tr \left[ \str{2.5ex} E(\bm \eta(x) ) \circ E(\bm \eta(x) ) \right] - 
\left[ \str{2.5ex} \bm \pi(x)^\tr E(\bm \eta(x) \right]^2
\eea
Again, $V(\eta_x)$ is a scalar.

\section{Results}
% You can do automatic table and fig referencing like this:
NEED TO ADD IN GRAPHS SHOWING THE WITHIN-GROUP COMPONENT. NEED TO ADD COMMENTS ON THE WITHIN-GROUP COMPONENT.
Table~\ref{tab:LESD0m} and Table~\ref{tab:LESD0f} show the life expectancy and
variation in age at death for males and females, respectively, in each
deprivation quintile at each Census year. The same tables reporting life expectancy and variation in age at death conditional upon survival to age 35 are included in the appendices.

% Table generated by Excel2LaTeX from sheet 'Sheet1'
\begin{table}[htbp]
  \centering
  \caption{Life expectancy and standard deviation for males, age 0.}
  \label{tab:LESD0m} % add this to give a label
    \begin{tabular}{lrrrrrrrr}
          & \multicolumn{2}{c}{1981} & \multicolumn{2}{c}{1991} & \multicolumn{2}{c}{2001} & \multicolumn{2}{c}{2011} \\
    \midrule
    quintile & \multicolumn{1}{c}{ex} & \multicolumn{1}{c}{sd} & \multicolumn{1}{c}{ex} & \multicolumn{1}{c}{sd} & \multicolumn{1}{c}{ex} & \multicolumn{1}{c}{sd} & \multicolumn{1}{c}{ex} & \multicolumn{1}{c}{sd} \\
    \midrule
    1 (least dep.) & 71.6  & 15.4  & 74.5  & 14.4  & 77.6  & 13.7  & 80.2  & 13.6 \\
    2     & 69.9  & 16.1  & 72.9  & 14.9  & 75.3  & 15.1  & 78.5  & 14.6 \\
    3     & 69.1  & 16.2  & 72.1  & 15.4  & 73.9  & 15.4  & 77.0  & 15.1 \\
    4     & 68.2  & 16.2  & 70.4  & 16.1  & 72.2  & 16.1  & 75.3  & 15.6 \\
    5 (most dep.) & 66.4  & 16.4  & 68.3  & 16.5  & 69.0  & 17.2  & 72.4  & 16.5 \\
    Total pop. & 69.0  & 16.1  & 71.6  & 15.6  & 73.5  & 15.8  & 76.7  & 15.4 \\
    \bottomrule
    \end{tabular}%
  \label{tab:addlabel}%
\end{table}%


% Table generated by Excel2LaTeX from sheet 'Sheet1'
\begin{table}[htbp]
  \centering
  \caption{Life expectancy and standard deviation for females, age 0.}
   \label{tab:LESD0f}
    \begin{tabular}{lrrrrrrrr}
          & \multicolumn{2}{c}{1981} & \multicolumn{2}{c}{1991} & \multicolumn{2}{c}{2001} & \multicolumn{2}{c}{2011} \\
    \midrule
    quintile & \multicolumn{1}{c}{ex} & \multicolumn{1}{c}{sd} & \multicolumn{1}{c}{ex} & \multicolumn{1}{c}{sd} & \multicolumn{1}{c}{ex} & \multicolumn{1}{c}{sd} & \multicolumn{1}{c}{ex} & \multicolumn{1}{c}{sd} \\
    \midrule
    1 (least dep.) & 77.1  & 14.9  & 79.1  & 14.1  & 81.3  & 13.0  & 83.4  & 12.8 \\
    2     & 75.8  & 15.3  & 78.6  & 13.9  & 80.4  & 13.6  & 82.1  & 13.5 \\
    3     & 75.1  & 15.6  & 77.7  & 14.7  & 78.9  & 14.5  & 80.9  & 13.9 \\
    4     & 74.4  & 15.7  & 76.6  & 14.8  & 78.1  & 14.6  & 80.0  & 14.0 \\
    5 (most dep.) & 72.8  & 16.3  & 74.9  & 15.9  & 76.3  & 15.7  & 77.9  & 15.1 \\
    Total pop. & 75.0  & 15.6  & 77.3  & 14.8  & 78.9  & 14.4  & 80.9  & 14.0 \\
    \bottomrule
    \end{tabular}%
  \label{tab:addlabel}%
\end{table}%

The most deprived quintile experiences the lowest life expectancy and
the highest variation in age at death (highest standard deviation) at each year. For males there was an
increase in variation in age at death between 1991 and 2001. Although there was
some improvement between 2001 and 2011, variation in age at death was very
similar to the level experienced 30 years earlier. Females from the most
deprived quintile have experienced decreasing variation in age at death (decreasing standard deviation) but the
decrease was greater for the least deprived. The change in standard
deviation over time, across all ages is shown in Figure ~\ref{fig:sdtotal}. Male
total variation has increased somewhat, and female total variation has changed
very little over the time period studied.

Figure~\ref{fig:decompbtwn} shows the proportion of the total difference in
variation in age at death that is due to within-group variance and that which is due to between-group inequality (CHANGE FIGURE TO ALSO INCLUDE WITHIN-GROUP COMPONENT). For males and females the proportion of variation explained by between group inequality has increased over time. The increase in this component was greater in magnitude for males than for females.

\begin{figure*}[t!]
    \centering
      \caption{Standard deviation of remaining lifespan for total population by age, Census years
      1981 until 2011.}
      \label{fig:sdtotal}
    \begin{subfigure}[t]{0.5\textwidth}
        \centering
        \caption{Males}
        \includegraphics[width=\textwidth]{Figures/TotalsdMales.pdf}
    \end{subfigure}%
    ~ 
    \begin{subfigure}[t]{0.5\textwidth}
        \centering
        \caption{Females}
        \includegraphics[width=\textwidth]{Figures/TotalsdFemales.pdf}
    \end{subfigure}
\end{figure*}

\begin{figure*}[t!]
    \centering
      \caption{Proportion of variance due to differences between deprivation
      quintiles by age, Census years 1981 until 2011.}
      \label{fig:decompbtwn}
    \begin{subfigure}[t]{0.5\textwidth}
        \centering
        \caption{Males}
        \includegraphics[width=\textwidth]{Figures/BetweenPropMales.pdf}
    \end{subfigure}%
    ~ 
    \begin{subfigure}[t]{0.5\textwidth}
        \centering
        \caption{Females}
        \includegraphics[width=\textwidth]{Figures/BetweenPropFemales.pdf}
    \end{subfigure}
\end{figure*}

\FloatBarrier
\subsection{Sensitivity analysis}
We tested the sensitivity of our results to the size of deprivation group by
replicating the analysis using deciles of deprivation, each representing 10\% of
the population. The conclusions were the same for males and for females.
However, the increase in the between group component over time was greater in
magnitude when using deciles. We chose to report results for quintiles of
deprivation as they are the preferred analytical grouping for routine reporting
of health measures in Scotland \citep{Health2017}.

\section{Discussion and conclusion}

\subsection{Summary of main findings}
Deprivation differences in age at death were evident at all Census years when measuring socioeconomic inequality by area-level. Those living in the most deprived areas can expect to live the shortest lives and experience the greatest variation in age at death: a double burden of mortality inequality. The difference between deprivation groups was larger for males than for females. Males from the most deprived quintile experienced increasing variation in age at death between 1991 and 2001 so that the level of variation in 2011 was the same as that experienced 30 years earlier. The between group component of inequality increased over the time period while the within group component stayed constant.

\subsection{Theoretical reflection - THIS SUB-SECTION IS WEAK AND I NEED TO THINK ABOUT WHAT I WANT TO SAY SOME MORE}
Although our results are unable to determine the exact reason why between group differences have become a relatively more important factor when accounting for variance in age at death at the population level they may provide further support for existing theories which emphasizes the role of relative deprivation and  social stratification for population level health and mortality outcomes \citep{Wilkinson2007,Marmot2001}. The timing of the increasing between group component may be associated with the well documented 'polarization' of deprivation, health and mortality that increased in the UK following the 1980s \citep{Shaw2000,Mitchell2000}. Therefore our results are important for governments to consider when deciding how best to tackle inequalities in age at death: whether to allocate resources to social policies that intervene at the contextual and area level versus social policies that intervene at the individual level  \citep{Allik2016,Roux2001,Robert1999,TheScottishGovernment2016}. In addition to the theoretical contribution, our study demonstrates a number of empirical strengths.   

\subsection{Study strengths and limitations}
The data used for this study includes the most robust population estimates and mortality data for the entire population of Scotland. Using a validated area-level measure of socioeconomic inequality meant that complete lifetables could be constructed and no ages were truncated from the analysis. However, it is important to acknowledge the reasons why studies interested in the social distribution mechanisms of adult mortality may consider restricting analysis to older ages. \citet{Smits2009} suggest only looking at ages 15+ because these are the ages where 80\% of deaths in developed countries now occur. Looking only at adult mortality may better reflect the causes of death driving mortality change in more recent time periods: infectious disease and effective medical intervention historically reduced infant and childhood deaths rapidly but reductions in adult mortality are influenced by more complex mechanisms that change slowly \citep{Smits2009,Vallin2004}. Our results indicate that the age at which the difference in variation in age at death is greatest is around 35 years old. This provides some reassurance for studies that are forced to truncate out younger age groups: the peak of variation in age at death (at least in developed countries) is likely to be captured. 

We recognize that our results are vulnerable to the ecological fallacy: it is possible that the association found at the area-level may differ from the association found at the individual \citep{Diez2002}. The consistency of our findings with the existing literature on socioeconomic inequalities in variation in age at death \citep{Broennum-Hansen2017,Raalte2014} indicates that the findings by area-level deprivation are not an artefact. This does not mean that area-level measures and individual level measures are substitutes for one another. Area-level measures ‘capture characteristics of populations’ and individual level measures ‘capture characteristics of individuals’ \citep{Leyland2007}. An example helps to illustrate the contentions. GPs aiming to reduce inequalities between individuals by providing preventative screening programmes may rely on area-level indicators to target those who are most deprived but may target an individual in a deprived area who is actually affluent. So relying on an area-level measure to reduce health inequalities between individuals can be problematic if there is an overriding assumption that the underlying characteristics of the population are socially homogenous \citep{Fischbacher2014}. We acknowledge that deprived individuals do not exclusively reside in deprived areas and affluent individuals do not exclusively reside in affluent areas \citep{Leyland2007}.

The Carstairs score as an empirical measure has been the focus of further criticisms. For example, the meaning of car ownership is fundamentally different for individuals in rural contexts compared to urban contexts. It is also acknowledged that overcrowding may occur out of choice and for cultural reasons rather than simply being a marker of deprivation \citep{Fischbacher2014}. Therefore it has been suggested that the Carstairs score may be an out of date measure of socioeconomic deprivation \citep{Schofield2016,Tunstall2011} because the relevance of the variables used for capturing the meaning of deprivation varies across contexts and over time \citep{Norman2010}. In response, it was demonstrated that the scores for each postcode sector at each Census year are highly correlated despite changes to the formal definitions of the variables. This is interpreted as evidence that the underlying information the variables aim to capture is similar or that deprivation has remained stable over time \citep{Leyland2007}. Alternative measures of area-level deprivation are available, for example the Scottish index of multiple deprivation (SIMD). The SIMD includes 38 indicators from 7 domains (employment,income,health,education,access to services, crime and housing). The SIMD was unsuitable for the trend focus of this research because it is only recommended for analysis using data beginning in 1996 \citep{Health2017}. A further limitation is that it includes indicators of health and mortality meaning that the full SIMD can not be used for health inequalities research. Instead health inequalities research tends to use the income domain only \citep{Leyland2007a}. The income domain is highly correlated with the full SIMD and is one of the most heavily weighted domains \citep{Health2017,TheScottishGovernment2016}.   

\section{Conclusion}
This study has demonstrated increasing contributions from area level deprivation differences to total variance in age at death using population level data for Scotland. This type of trend analysis is important for understanding the changing nature of mortality inequalities in developed countries. Monitoring variance in age at death is complimentary to the routine monitoring of life expectancy: monitoring both allows us to establish if average population mortality and mortality inequalities have been improved simultaneously. More countries should begin to measure the between group and within group contributions to variance in age at death and monitor trends in order to understand the extent to which mortality is dependent upon, and amenable to, area level deprivation.


\bibliographystyle{apalike}
\bibliography{references}


\section{Appendices}
% Table generated by Excel2LaTeX from sheet 'Sheet1'
\begin{table}[htbp]
  \centering
  \caption{Life expectancy and standard deviation for males, age 35.}
    \begin{tabular}{lrrrrrrrr}
          & \multicolumn{2}{c}{1981} & \multicolumn{2}{c}{1991} & \multicolumn{2}{c}{2001} & \multicolumn{2}{c}{2011} \\
    \midrule
    quintile & \multicolumn{1}{c}{ex} & \multicolumn{1}{c}{sd} & \multicolumn{1}{c}{ex} & \multicolumn{1}{c}{sd} & \multicolumn{1}{c}{ex} & \multicolumn{1}{c}{sd} & \multicolumn{1}{c}{ex} & \multicolumn{1}{c}{sd} \\
    \midrule
    1 (least dep.) & 38.4  & 11.4  & 40.9  & 11.2  & 43.7  & 11.3  & 46.3  & 11.1 \\
    2     & 37.1  & 11.6  & 39.5  & 11.6  & 41.9  & 11.8  & 44.7  & 12.0 \\
    3     & 36.4  & 11.8  & 38.8  & 11.8  & 40.5  & 12.2  & 43.4  & 12.4 \\
    4     & 35.5  & 11.8  & 37.6  & 11.9  & 39.1  & 12.5  & 41.7  & 13.0 \\
    5 (most dep.) & 33.8  & 12.1  & 35.8  & 12.3  & 36.6  & 13.2  & 39.2  & 13.5 \\
    Total pop. & 36.2  & 11.8  & 38.5  & 11.8  & 40.3  & 12.4  & 43.1  & 12.6 \\
    \bottomrule
    \end{tabular}%
  \label{tab:addlabel}%
\end{table}%

% Table generated by Excel2LaTeX from sheet 'Sheet1'
\begin{table}[htbp]
  \centering
  \caption{Life expectancy and standard deviation for females, age 35.}
    \begin{tabular}{lrrrrrrrr}
          & \multicolumn{2}{c}{1981} & \multicolumn{2}{c}{1991} & \multicolumn{2}{c}{2001} & \multicolumn{2}{c}{2011} \\
    \midrule
    quintile & \multicolumn{1}{c}{ex} & \multicolumn{1}{c}{sd} & \multicolumn{1}{c}{ex} & \multicolumn{1}{c}{sd} & \multicolumn{1}{c}{ex} & \multicolumn{1}{c}{sd} & \multicolumn{1}{c}{ex} & \multicolumn{1}{c}{sd} \\
    \midrule
    1 (least dep.) & 43.4  & 11.7  & 45.1  & 11.5  & 47.0  & 11.1  & 49.0  & 49.0 \\
    2     & 42.2  & 12.0  & 44.4  & 11.7  & 46.1  & 11.5  & 47.9  & 47.9 \\
    3     & 41.6  & 12.2  & 43.8  & 12.1  & 45.0  & 12.0  & 46.7  & 46.7 \\
    4     & 41.0  & 12.1  & 42.7  & 12.3  & 44.1  & 12.2  & 45.7  & 45.7 \\
    5 (most dep.) & 39.6  & 12.7  & 41.4  & 12.8  & 42.6  & 13.1  & 43.9  & 43.9 \\
    Total pop. & 41.5  & 12.2  & 43.4  & 12.2  & 44.9  & 12.1  & 46.7  & 46.7 \\
    \bottomrule
    \end{tabular}%
  \label{tab:addlabel}%
\end{table}%


\end{document}