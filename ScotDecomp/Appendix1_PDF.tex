\documentclass[12pt,oneside,a4paper]{article} % for sharing
\usepackage{apacite}
\usepackage{appendix}
\usepackage{amsmath}
\usepackage{amsthm}
\usepackage{multirow}
\usepackage{amssymb} % for approx greater than
\usepackage{caption}
\usepackage{placeins} % for \FloatBarrier
\usepackage{graphicx}
\usepackage{subcaption}
\usepackage{longtable}
\usepackage{setspace}
\usepackage{booktabs}
\usepackage{tabularx}
\usepackage{xcolor,colortbl}
\usepackage{chngpage}
\usepackage{natbib}
\bibpunct{(}{)}{,}{a}{}{;} 
\usepackage{url}
\usepackage{nth}
\usepackage{authblk}
\usepackage[most]{tcolorbox}
\usepackage[normalem]{ulem}
\usepackage{amsfonts}

% scraped out necessary stuff from hal's loghead file

% new commands defined

%bold faced letters
\newcommand{\bo}[1]{{\bf #1}}

\newcommand{\bm}[1]{\mbox{\boldmath $#1$}}
\newcommand{\kron}{\otimes}
\renewcommand{\vec}{\mbox{vec} \,}

%begin and end equations
\newcommand{\be}{\begin{equation}}
\newcommand{\ee}{\end{equation}}

%begin and end equations without numbers
\newcommand{\bd}{\begin{displaymath}}
\newcommand{\ed}{\end{displaymath}}

%begin and end equation arrays
\newcommand{\bea}{\begin{eqnarray}}
\newcommand{\eea}{\end{eqnarray}}

%begin and end equation arrays without numbers
\newcommand{\beastar}{\begin{eqnarray*}}
\newcommand{\eeastar}{\end{eqnarray*}}

%begin and end matrices
\newcommand{\bmat}[1]{\left(\begin{array}{#1}}
\newcommand{\emat}{\end{array}\right)}

%equation numbers
\newcommand{\enum}[1]{\label{eq:#1}}

%derivatives and elasticities
\newcommand{\der}[2]{{d #1 \over d #2}}
\newcommand{\elas}[2]{{\varepsilon #1 \over \varepsilon #2}}


%partial derivatives and second partial derivatives
\newcommand{\pder}[2]{{\partial #1 \over \partial #2}}
\newcommand{\secpderij}[3]{{\partial^2 #1 \over \partial #2 
            \partial #3}}
\newcommand{\tsecpderij}[3]{\partial^2 #1 / \partial #2 
            \partial #3}  %text second partial

\newcommand{\secpderii}[2]{{\partial^2 #1 \over \partial #2^2}}
\newcommand{\tsecpderii}[2]{\partial^2 #1 / \partial #2^2}
%   text second partial

%matrix transpose symbol
\newcommand{\tr}{{\mbox{\tiny \sf T}}}

%matrix diagonal symbol
%\newcommand{\diag}{\mbox{diag} \,}
\newcommand{\diag}{\mathcal{D}}

\newcommand{\trace}{\mbox{trace}}

\newcommand{\dg}{{\rm{dg}}}

% \mc{number of columns}{position}{item}
\newcommand{\mc}[3]{\multicolumn{#1}{#2}{#3}}
% center a single item in a table
\newcommand{\cent}[1]{\mc{1}{c}{#1}}

\newcommand{\mcal}[1]{\mathcal{#1}}


\newcommand{\str}[1]{\rule{0in}{#1}}
\newcommand{\tbar}[1]{\left. \rule{0in}{#1} \right|}





\newcommand{\rind}{\noindent\hangindent=20pt\hangafter=1}

%%%%%%%%%
% margin adjustments
% these are for US letter size paper; should be adjusted for A4
%%%%%%%%

\textwidth=6.5in
 \oddsidemargin=0in
 %\evensidemargin=0.125in
 %\evensidemargin=0.5in
 \textheight=9.0in
 \topmargin=0.0in
 \headheight=0.0in
 \headsep=0.0in

%%%%%%%%%%%%
% end of margin adjustments
%%%%%%%%%%%% 

%%%%%% stuff from here on not used much %%%%%

% \newtheorem{theorem}{Theorem}[section]
%\newtheorem{theorem}{Theorem}

%\newtheorem{example}{Example}[section]
\newtheorem{example}{Example}

    \newcommand{\exbegin}[1]{\begin{itemize} \item[~]

    \begin{example}{\bf #1} \begin{rm} }
    \newcommand{\exend}{\end{rm}\end{example}
    \end{itemize}}
\newcommand{\scp}[1]{\langle #1 \rangle}
\setcounter{secnumdepth}{3}
\setcounter{tocdepth}{4}

\newtheorem{newfig}{Figure}
\newcommand{\figbegin}{\begin{newfig} \begin{rm} }
\newcommand{\figend}{\end{rm} \end{newfig}}
% commands to change float restrictions
\renewcommand{\topfraction}{0.99}
\renewcommand{\bottomfraction}{0.99}
\renewcommand{\textfraction}{0}
\renewcommand{\floatpagefraction}{0}

\newcommand{\mathbold}[1]{\mbox{\boldmath $\bf#1$}}
% Shultis, Latex Notes, p. 54


%Commands for biblist
\newenvironment{biblist}{\begin{list}{}{\setlength{\leftmargin}{0.15in}% 
\setlength{\itemindent}{-0.15in} \setlength{\parsep}{0.in}%
\setlength{\itemsep}{0.1in}}}{\end{list}} 

% columns for longtable
\newcolumntype{C}[1]{>{\centering\let\newline\\\arraybackslash\hspace{0pt}}m{#1}}
\newcolumntype{L}[1]{>{\raggedright\let\newline\\\arraybackslash\hspace{0pt}}m{#1}}
\usepackage{arydshln} % Dashed lines in matrices


\usepackage[margin=1in]{geometry}
%\doublespacing % for review

% line numbers to make review easier
%\usepackage{lineno}
%\linenumbers

%\usepackage{soul}% for \st{}

%%%%%%%%%%%%%%%%%%%%%%%%%%%%%%%%%%%%%%%%%%%%%%%%%%%%%%%%%%%%%%%%%%%%%%%%%%%%%%
% for section 4 math environments
\theoremstyle{definition}
\newtheorem{definition}{Definition}[section]
\newtheorem{theorem}{Theorem}[section]
\newtheorem{proposition}{Proposition}[section]
\newtheorem{corollary}{Corollary}[proposition]
\newtheorem{remark}{Remark}[section]

%%%%%%%%%%%%%%%%%%%%%%%%%%%%%%%%%%%%%%%%%%%%%%%%%%%%%%%%%%%%%%%%%%%%%%%%%%%%%%

\newcommand\ackn[1]{%
  \begingroup
  \renewcommand\thefootnote{}\footnote{#1}%
  \addtocounter{footnote}{-1}%
  \endgroup
}

% Affiliations in small font size
\renewcommand\Affilfont{\small}

\defcitealias{HMD}{HMD 2016}

% junk for longtable caption
\AtBeginEnvironment{longtable}{\linespread{1}\selectfont}
\setlength{\LTcapwidth}{\linewidth}

% sort van Raalte properly
% #1: sorting key, #2: prefix for citation, #3: prefix for bibliography
\DeclareRobustCommand{\VAN}[3]{#2} % set up for citation

%%%%%%%%%%%%%%%%%%%%%%%%%%%%%%%
\begin{document}

\section{Detailed description of decomposition method}

% Detailed description of methods moved from main body of text
From the lifetable, we extract
conditional single-age death probabilities, $q_x$, and take its complement,
$p_x$. We then calculate the survival matrix for the $i^{th}$
quintile, $\bo U_i$ as:
\begin{equation}
\bo U_i = 
\begin{bmatrix}
    0     & \hdots  & \hdots &  \hdots  & 0 \\
    p_{1} &   &    &    &  \vdots \\
    0 & \ddots &   &   & \vdots \\
    \vdots & & \ddots & & 0\\
   0 &  \hdots & 0 & p_{\omega-1}  & p_{\omega}
\end{bmatrix}
\end{equation}
Conditional remaining survivorship is calculated as:
\begin{equation}
\mathbf{N}_i = (\mathbf{I} - \mathbf{U}_i )^{-1} \quad .
\end{equation}
$\mathbf{N}_i$ ends up being 0s in the upper triangle, and conditional remaining
survivorship in columns descending from the subdiagonal. The moments of longevity for individuals in group $i$ are $\bm \eta_1^{(i)}$ and $\bm \eta_2^{(i)}$. 
\begin{equation}
\bm \eta_1^{(i)} = (1^\tr \bm N_i)^\tr
\end{equation}
The second moment is defined as:
\begin{equation}
\bm \eta_2^{(i)} = \left[ 1^\tr \bm N_i (2 \bm N_i - \bm I)\right]^\tr
\end{equation}
The vectors with the means and variances, for group $i$, are
\bea
E(\bm \eta^{(i)}) &=& \bm \eta_1^{(i)} \\
V(\bm \eta^{(i)}) &=& \bm \eta_2^{(i)} - \left[ \bm \eta_1^{(i)} \ \circ \bm \eta_1^{(i)} \right]
\eea

To carry out calculations we procede by creating vectors that contain the
combined age and stage specific values
\bea
E(\tilde{\bm \eta}) = \bmat{c}
E(\bm \eta^{(1)}) \\
\vdots\\
E(\bm \eta^{(g)})
\emat
\eea
and a similar vector for variances $V(\tilde{\bm \eta})$. The tilde indicates
that these combine both age and quintile values, with length = $g \omega$.

The next step is to calculate the means and variances of remaining longevity, at
each age $x$, within each group, as follows.
\bea
E(\bm \eta(x)) &=& \left( \bo I_g \kron \bo e_x^\tr \right) E(\tilde{\bm \eta}) \qquad x=1,\ldots,\omega  \\[1ex]
V(\bm \eta(x)) &=& \left( \bo I_g \kron \bo e_x^\tr \right) V(\tilde{\bm \eta}) \qquad x=1,\ldots,\omega
\eea
where $\bo e_x$ is a vector of length $\omega$ with a 1 in the $x$th position and zeros elsewhere. The resulting vectors here are of dimension $g \times 1$.

At age $x$ the cohort consists of a mixture of the $g$ different groups ($g=5$
for quintiles, 10 for deciles) with mixing distribution
$\bm \pi(x)$ generated by the differential survival of groups within the cohort.


The mixing distribution \bm \pi(x) at age $x$ is a vector of dimension $g\times 1$ , which sums to 1. It is obtained from the distribution of groups in an initial cohort. Since quintiles are by definition equally distributed, it would seem that the initial cohort should be evenly distributed. Some other distribution could be used if desired. 

Let \bm \pi(0) be the initial mixing distribution, and let $\bm \eta^{(i)}(0)$ be the initial cohort age distribution in group $i$. Then 
\begin{equation}
\bo n^{(i)} (0) = \bo{e}_i \pi_i(0)
\end{equation}
(i.e., a vector with $\pi_i (0)$ in the first entry and zeros elsewhere. For an evenly distributed cohort, $\pi_i (0)=1/g$.) 

Project each group with its appropriate survival matrix 
\begin{equation}
\bo n^{(i)} (x) = \bo U_i^{x}\bo n^{(i)}(0)  \qquad i=1,\ldots,g\
\end{equation}

add up the entries
\begin{equation}
N^{(i)}(x)=\bo 1_{\omega}^{\tr}\bo n^{(i)}(x) \qquad i=1,\ldots,g\
\end{equation}

and create $\bm \pi(x)$ by putting these into a vector and normalizing it to sum to 1
\bea
\bm \pi(x) = \bmat{c}
N^{(1)}(x) \\
\vdots\\
N^{(g)}(x)
\emat
\frac{1}{\sum_{i}N(i)(x)}
\eea


At age $x$ remaining life expectancy for the total population is
\bea
E (\eta_x) &=& E_{\bm \pi(x)} \left[ E(\bm \eta(x)) \right] \\[1ex]
&=& \bm \pi(x)^\tr E(\bm \eta(x)) \\[1ex]
&=& \left( \bm \pi(x)^\tr \kron \bo e_x \right) E(\tilde{\bm \eta}) \qquad x=1,\ldots,\omega
\eea
Notice that $\eta_x$ is a scalar. The remaining life expectancy at age $x$ is a
simple average weighted by the mixing distribution.

The variance in $\eta_x$ is
\be
V(\eta_x) = V_{\rm within} + V_{\rm between}
\ee
with
\bea 
V_{\rm within} &=& E_{\bm \pi(x)} \left[ \str{2.5ex}V \left( \bm \eta(x) \right) \right]  \\[1ex]
&=&
\bm \pi(x)^\tr V(\bm \eta(x) )\\[1ex]
&=& \left( \bm \pi(x)^\tr \kron \bo e_x^\tr \right) V(\tilde{\bm \eta}(x)) 
\eea
and
\bea
V_{\rm between} &=& V_{\bm \pi(x)} \left[ \str{2.5ex} E \left( \bm \eta(x) \right) \right] \\
&=&
\bm \pi(x)^\tr \left[ \str{2.5ex} E(\bm \eta(x) ) \circ E(\bm \eta(x) ) \right] - 
\left[ \str{2.5ex} \bm \pi(x)^\tr E(\bm \eta(x) \right]^2
\eea
Again, $V(\eta_x)$ is a scalar.\end{document}